\documentclass[9pt]{sig-alternate-05-2015}

\usepackage{bm,amsmath}
\usepackage{subfigure,epsfig,graphicx}
\usepackage{verbatim}
\usepackage{cite}
\usepackage{url}
\usepackage{algorithm}
\usepackage{algorithmic}
\usepackage{color}
\usepackage{multirow}
\usepackage{caption} 
\captionsetup[table]{skip=2pt}




\begin{document}

% Copyright
\setcopyright{acmcopyright}

\title{Translating on Pairwise Entity Space for Multi-Relational Data Embedding}

%
% --- Author Metadata here ---
\numberofauthors{3}
% 1st,author
\author{
\alignauthor
Yu Wu \\
       \affaddr{The University of Liverpool}\\
       \affaddr{Brownlow Hill}\\
       \affaddr{ Liverpool, UK}\\
       \email{yuwu@liverpool.ac.uk}
 % 2nd. author
\alignauthor
Tingting Mu \\
       \affaddr{The University of Liverpool}\\
        \affaddr{Brownlow Hill}\\
       \affaddr{Liverpool, UK}\\
       \email{t.mu@liverpool.ac.uk}
 % 3rd. author
 \alignauthor
John Yannis Goulermas \\
       \affaddr{The University of Liverpool}\\
        \affaddr{Brownlow Hill}\\
       \affaddr{Liverpool, UK}\\
       \email{j.y.goulermas@liverpool.ac.uk}
 }
\date{11 Feb 2016}
             
\maketitle

\begin{abstract}
In this paper, we have studied the problem of embedding entity and relations of the multi-relational data into one unified vector space which preserves certain information within the data and can then be used to predict the new relations between the entities. Inspired by the one existing work, \emph{TransE} \cite{bordes_translating_2013}, we have proposed an algorithm which models each fact by treating the projected relation type vector as a translation operation on the pairwise entity space. We have performed the link prediction task on two large benchmark knowledge base dataset. The experiment show that the proposed algorithm has posed a significant improvement over  \emph{TransE}.
\end{abstract}

%
% The code below should be generated by the tool at
% http://dl.acm.org/ccs.cfm
% Please copy and paste the code instead of the example below. 
%

\begin{CCSXML}
<ccs2012>
<concept>
<concept_id>10010147.10010257.10010293.10010297.10010299</concept_id>
<concept_desc>Computing methodologies~Statistical relational learning</concept_desc>
<concept_significance>500</concept_significance>
</concept>
<concept>
<concept_id>10010147.10010257.10010293.10010319</concept_id>
<concept_desc>Computing methodologies~Learning latent representations</concept_desc>
<concept_significance>500</concept_significance>
</concept>
</ccs2012>
\end{CCSXML}

\ccsdesc[500]{Computing methodologies~Statistical relational learning}
\ccsdesc[500]{Computing methodologies~Learning latent representations}


%
% End generated code
%

%
%  Use this command to print the description
%
\printccsdesc

% keyword not necessary
% \keywords{Knowledge Graphs;}


\section{Introduction}
Traditional machine learning algorithms usually works with data that are described by its feature vector of attributes. While statistical relational learning handles the data where objects are interlinked by various relation types. Thus, the multi-relational data can be referred as in the form of \emph{graph} whose nodes represent the entities and edges corresponds to the relationships. In particular, we consider to apply relational learning algorithms to the large-scale \emph{knowledge bases}, which encode real-world information by representing each fact following the Resource Description Framework (RDF) format.  Examples of knowledge bases include Wordnet \cite{miller_wordnet:_1995}, YAGO \cite{suchanek_yago:_2007}, DBpedia \cite{lehmann_dbpedialarge-scale_2015}, Freebase \cite{bollacker_freebase:_2008}, NELL \cite{betteridge_toward_2009} and the Google's Knowledge Vault project\cite{dong_knowledge_2014}.  These knowledge bases are very useful for information retrieval, question answering and other related AI tasks.

The knowledge bases are very large and usually contain millions of nodes and billions of edges. They can be very useful for information retrieval, question answering and other related AI tasks. In fact, most databases are noisy and far from complete as such large databases are either constructed collaboratively or (partly) automatically online. Typical tasks of relational learning on knowledge bases include correcting the missing or invalid relations, identify the duplicate entities and link-based clustering. 

To work with these huge knowledge databases, there has arisen an effective approach which embedding the entities and relationships into a continuous vector space while the knowledge graph is modelled by subsequent geometric operations. Among them, \emph{TransE} \cite{bordes_translating_2013} is a highly promising model which has achieved very well performance with the minimal parametrisation. However, the assumption behind \emph{TransE} is too simple to adequately model the knowledge graphs. Recently, various other algorithms \cite{fan_transition-based_2014} \cite{wang_knowledge_2014} \cite{lin_learning_2015} \cite{garcia-duran_composing_????}  have been proposed to extend the \emph{TransE} model to explore for a more robust assumption. These algorithms have either extended the \emph{TransE} by   weighting adjustment scheme base on each relation types  or further making assumption that the interaction could only happen on the respective relation space for each relationship. 

In this paper, we have proposed a method called Translating on Pairwise Entity Space(denoted as \emph{TranPES}),  which has put the entities and relationships into a unified vector space while only the relation vector's projection on the pairwise entity space get involved for modelling each entity pair related fact in the knowledge graphs. It has then been trained on a ranking objective using stochastic gradient decsent and we have compare it with other methods on the two commonly used data  of WordNet \cite{miller_wordnet:_1995} and Freebase \cite{bollacker_freebase:_2008} on the link prediction task.

The remainder of this paper is organised as follow. In Section \ref{review}, we briefly review the previous work which following the knowledge graph embedding approach. The mathematical formulation for our model and the related discussion is presented in Section \ref{math}. Subsequent experiments are conducted in Section \ref{exp}. Finally, we draw the conclusion and discuss about the future direction in the last Section.

\section{Related work} \label{review}
Embedding knowledge bases into a distributed vector space have gained a great interest recently. Embedding-based model usually design various scoring function to measure the plausibility of each observable fact in the knowledge graph. And all models explain the confidence score via the assumed embeddings of entities. Indeed, this idea of learning word embeddings  have been shown very successful in the natural language processing that such representations can store key information for each word and help to improve the performance on standard NLP tasks \cite{bengio_neural_2006}.  Depending on the representation learning methods, the embedding-based model can be classified into tensor/matrix factorisation \cite{singh_relational_2008} \cite{nickel_three-way_2011}, Bayesian clustering framework \cite{kemp_learning_2006} \cite{sutskever_modelling_2009} and neural networks \cite{bordes_learning_2011} \cite{jenatton_latent_2012} \cite{bordes_translating_2013} \cite{socher_reasoning_2013} \cite{bordes_semantic_2014} \cite{wang_knowledge_2014} \cite{lin_learning_2015} \cite{garcia-duran_composing_????}. We focusing on the neural network model as they are highly scalable and in particular the \emph{TransE} \cite{bordes_translating_2013} algorithm. The \emph{TransE} algorithm makes a simple assumption that regards the relation type as a translation operation on the embedded entity vectors. Such a simple assumption has led to the \emph{TransE} algorithm significantly outperforms state-of-art methods on link prediction tasks. A lot of works \cite{fan_transition-based_2014} \cite{wang_knowledge_2014} \cite{lin_learning_2015} \cite{garcia-duran_composing_????}  based on \emph{TransE} have been proposed and have gained better performance. By taking the advantage of the translation operation introduced by \emph{TransE}, we propose a novel approach that limit the translation operation conducted only on the entity pair hyperplane and tries to learn represent the different features of relationships into different subspaces.





\section{Proposed Model} \label{math}
The knowledge graph data $\mathcal{D}$ consists of a set of links between a fixed set of entities. Let $\mathcal{E} = \{e_1, \ldots, e_{N_e}\}$ denote the entity set and $\mathcal{R} = \{r_1, \ldots, r_{N_r}\}$ the link set.  The relationship between entities indicated by $\mathcal{D}$ are represented as relation triples such as $(e_h,r_{\ell},e_t)$, where $e_h, e_t\in \mathcal{E}$ are referred as the head  and tail, respectively, and  $r_{\ell} \in \mathcal{R}$ the link.  For example, $(\emph{Champa}, \emph{formOfgoverment}, \emph{Monarchy})$ is one of such relation triples which the head entity \emph{Champa} and the tail entity \emph{Monarchy} is linked by the relation type \emph{formOfgoverment}.  For the sake of convenience of discussion, we simplified the relation triple as $(h,\ell, t)$   by referring only the indices of the entities and links. Given a set of known links within $\mathcal{D}$, the goal of a relational learning task is to  infer unknown links and correct the known links to complete the graph $\mathcal{D}$. One way to solve this task is to learn an energy function $E(h, \ell, t)$ on the set of all possible triples $\mathcal{E} \times \mathcal{R} \times \mathcal{E}$ so that a triple representing a truly existing link between two entities  is assembled with a low energy, otherwise with high energy. 

%For instance, $E(h, l, t)\geq c$ indicates there exists link $r_l$ between the entities $e_h$ and $e_t$ and $E(h, l, t)<c$ otherwise, where $c$ is a threshold. 

\subsection{Energy Function}
The proposed method  \emph{\emph{TranPES}}  parameterises the energy function of an input relation triple over three individual $k$-dimensional embedding vectors of its head, tail and link, as well as as a set of $k\times k$ transformation matrices $\{\textbf{P}_{ht}\}_{h,t}$ where different matrices are constructed for different head-tail entity pairs $(h,t)$. Letting $\{\bm e_i\}_{i=1}^{N_e}$ denote the entity embedding vectors and  $\{\bm r_i\}_{i=1}^{N_r}$ the link embeddings, the \emph{\emph{TranPES}} energy function is defined as the following:
\begin{equation} \label{energy}
\mathrm{E}{(h,\ell, t)} = \|\bm{e}_h +\textbf{P}_{ht} \bm{r}_{\ell}-\bm{e}_t\|_2.
\end{equation}
Apart from the $l_2$-norm, it is  possible to compute the energy based on other norms, e.g., the $l_1$-norm, $p$-norm, or other dissimilarity measures. Instead of freely setting the transformation matrices, $\textbf{P}_{ht}$ is restricted as a projection matrix that  gives the projection of a $k$-dimensional vector onto the space spanned by the two $k$-dimensional entity vectors $\bm{e}_h$ and $\bm{e}_t$.  Letting the columns of the $k\times 2$ matrix $\textbf{E}_{ht}$ store the two entity embedding vectors $\bm{e}_h $ and $\bm{e}_t$, the projection matrix is given by 
\begin{equation}
\textbf{P}_{ht}=\textbf{E}_{ht}\left(\textbf{E}_{ht}^T \textbf{E}_{ht} \right)^{-1}\textbf{E}_{ht}^T.
\end{equation}
To deal with cases when $\bm{e}_h$ and $\bm{e}_t$ take the same direction, the above expression can be further modified by adding a  regularisation term, leading to
\begin{equation}
\textbf{P}_{ht}=\textbf{E}_{ht}\left(\textbf{E}_{ht}^T \textbf{E}_{ht} +\xi\textbf{I}\right)^{-1}\textbf{E}_{ht}^T.
\end{equation}
The orthogonal complementary $(\textbf{I} -  \textbf{P}_{ht}) \bm{r}_{\ell}$ of the projected link embedding vector  is always close to perpendicular to  $\bm{e}_h$ and $\bm{e}_t$. This can be shown by the following:
\begin{align}
 &\begin{bmatrix} \bm{e}_h^{T} \\ \bm{e}_t^{T} \end{bmatrix} (\textbf{I} -  \textbf{P}_{ht})\bm{r}_{\ell} \notag  \\
=&\textbf{E}_{ht}^{T}\left(\textbf{I} -  \textbf{P}_{ht}\right)\bm{r}_{\ell}  \notag \\
=&\textbf{E}_{ht}^{T}\left(\bm{r}_{\ell} -  \textbf{E}_{ht}\left(\textbf{E}_{ht}^{T} \textbf{E}_{ht} + \xi \textbf{I} \right)^{-1} \textbf{E}_{ht}^{T} \bm{r}_{\ell}\right)  \notag \\
=&\textbf{E}_{ht}^{T}\bm{r}_{\ell} - \left (\textbf{E}_{ht}^{T} \textbf{E}_{ht}+\xi \textbf{I} - \xi \textbf{I}\right)\left(\textbf{E}_{ht}^{T} \textbf{E}_{ht} + \xi \textbf{I}\right )^{-1} \textbf{E}_{ht}^{T} \bm{r}_{\ell} \notag \\
=& \textbf{E}_{ht}^{T}\bm{r}_{\ell} - \textbf{E}_{ht}^{T} \bm{r}_{\ell}  +  \xi \left(\textbf{E}_{ht}^{T} \textbf{E}_{ht} + \xi \textbf{I}\right )^{-1} \textbf{E}_{ht}^{T} \bm{r}_{\ell}  \notag \\
=& \xi \left(\textbf{E}_{ht}^{T} \textbf{E}_{ht} + \xi \textbf{I}\right )^{-1} \textbf{E}_{ht}^{T} \bm{r}_{\ell}   \label{temp}
\end{align}

Let the economic singular value decomposition of the $k \times 2$ matrix $\textbf{E}_{ht}$ as:
\begin{equation*}
\textbf{E}_{ht} = \textbf{U} \mathbf{\Sigma} \textbf{V}^{T}
\end{equation*}
where \textbf{U} is a $k \times 2$ real unitary matrix, $\mathbf{\Sigma}$ is a $2 \times 2$ rectangular diagonal matrix with non-negative real numbers on the diagonal, and \textbf{V} is a $2 \times 2$ real unitary matrix.

Then the Eqs. (\ref{temp}) can be further simplified as:
\begin{align}
& \xi \left(\textbf{E}_{ht}^{T} \textbf{E}_{ht} + \xi \textbf{I}\right )^{-1} \textbf{E}_{ht}^{T} \bm{r}_{\ell} \notag \\
=& \xi \left(\textbf{V} \mathbf{\Sigma} \textbf{U}^{T}  \textbf{U} \mathbf{\Sigma} \textbf{V}^{T} + \xi \textbf{I}\right )^{-1} \textbf{V} \mathbf{\Sigma} \textbf{U}^{T} \bm{r}_{\ell} \notag \\
=& \xi \left(\textbf{V} \mathbf{\Sigma}^2 \textbf{V}^{T} + \xi \textbf{I}\right )^{-1} \textbf{V} \mathbf{\Sigma} \textbf{U}^{T} \bm{r}_{\ell} \notag \\
=& \xi \textbf{V} \left(\mathbf{\Sigma}^2 + \xi \textbf{I} \right)^{-1} \textbf{V}^{T} \textbf{V} \mathbf{\Sigma} \textbf{U}^{T} \bm{r}_{\ell} \notag \\
=& \xi \textbf{V} \left(\mathbf{\Sigma}^2 + \xi \textbf{I} \right)^{-1} \mathbf{\Sigma} \textbf{U}^{T} \bm{r}_{\ell} \notag \\
=& \xi \left( \textbf{V} \begin{bmatrix} \frac{\sigma_1}{\sigma_1^2 + \xi} & 0 \\ 0 & \frac{\sigma_2}{\sigma_2^2 + \xi}   \end{bmatrix} \textbf{U}^{T} \bm{r}_{\ell}  \right)\notag \\
\approx & \bm{0}
\end{align}

Thus, the regularisation term $\xi \textbf{I}$ not only making the projection transformation matrices $\{\textbf{P}_{ht}\}_{h,t}$ calculable in all circumstances but also preserves the property that the projected link embedding vector  could only be spanned by the corresponding entity embedding vectors.

\subsection{Training}
Given a set of known links between entities, a set of valid triples can be created from it. This set is referred as the positive triple  set, denoted as $\mathcal{D}^+$. For each positive triple $(h,l,t)\in\mathcal{D}^+$, a set of  corrupted triples can be generated by replacing either its head or tail  with a different one, given as
\begin{align*}
\small
\mathcal{D}^{-}_{h,l,t} = &  \left\{ (h^{'},\ell,t)  |  h^{'} \in \{ 1,2,\ldots, N_e\}, (h^{'},\ell,t)\notin \mathcal{D}^+ \right\} \cup \\
 &  \left\{ (h,\ell,t^{'})  |   t^{'} \in \{ 1,2,\ldots, N_e\},  (h,\ell,t^{'}) \notin \mathcal{D}^+\right\}
\end{align*}

To learn the energy function in Eq. (\ref{energy}) parameterised on the entity and link embeddings is equivalent to the optimisation of these embedding vectors to encourage the maximum discrimination between  the positive  and negative triples. To achieve this, a margin-based ranking loss can be employed, given as
\begin{equation} \label{loss}
\small
\mathcal{L}_m = \sum_{(h,\ell,t)\in \mathcal{D}^+} \sum_{(h^{'},\ell,t^{'})\in  \mathcal{D}^{-}_{h,\ell,t}} [\gamma + \mathrm{E}(h,\ell,t)- \mathrm{E}(h^{'},\ell,t^{'}) ]_+,
\end{equation}
where $[*]_+$ denotes the positive part of the input number $*$, and $\gamma > 0$ is a user-set margin parameter.

A length constraint can be imposed to the entity embeddings to prevent the training process to trivially minimise $\mathcal{L}_m$ by artificially increasing the scale of the entity embedding. This constraint can be incorporated into the cost function $\mathcal{L}_m$.  And we add the regularisation term  for the relational embedding vectors. This leads to a regularised cost function given as
\begin{equation} \label{loss2}
\mathcal{L} = \mathcal{L}_m  +\lambda_1 \sum_{i =1}^{N_e} \left[\|\bm{e}_i\|_2^2 - 1\right]_+ + \lambda_2 \sum_{j=1}^{N_r} \| \bm{r}_{j} \|_2^2,
\end{equation}
where $\lambda_1, \lambda_2 >0$ are the regularisation parameters. Finally, the  following optimisation problem is solved, given as
\begin{equation}
\min_{\{\bm e_i\}_{i=1}^{N_e}, \{\bm r_i\}_{i=1}^{N_r}} \mathcal{L}\left(\{\bm e_i\}_{i=1}^{N_e}, \{\bm r_i\}_{i=1}^{N_r}, \xi,\lambda_1, \lambda_2 \right).
\end{equation}

Similar to the optimisation procedure used in \cite{bordes_translating_2013}, a stochastic gradient descent algorithm in minibatch mode is used. The embedding vectors are first initialised following the random procedure in \cite{glorot_understanding_2010}. The algorithm is stopped based on its performance on a validation set.  Pseudo code of the proposed algorithm is provided in Algorithm $\textbf{1}$.


\begin{algorithm*}
\caption{Learning \emph{TranPES} }
\textbf{input} Training set $\mathcal{D} = \{(h,\ell,t)\}$, entities and rel. set $\mathcal{E}$ and $\mathcal{R}$, margin $\gamma$, regularisation term $\lambda_1, \lambda_2$ and embeddings dimension $k$.
\begin{enumerate}
\item \textbf{initialise} 
\begin{description}
\item $\bm{r} \leftarrow$ uniform $(-\frac{6}{\sqrt{k}}, \frac{6}{k})$ for each $\bm{r} \in \mathcal{R}$ 
\item	$\bm{r} \leftarrow \bm{r} / \| \bm{r} \|$ for each $\bm{r} \in \mathcal{R}$
\item $\bm{e} \leftarrow$ uniform $(-\frac{6}{\sqrt{k}}, \frac{6}{k})$ for each $\bm{e} \in \mathcal{E}$ 
\item	$\bm{e} \leftarrow \bm{e} / \| \bm{e} \|$ for each $\bm{e} \in \mathcal{E}$
\end{description}

\item \textbf{loop}
\begin{description}
\item $\mathcal{D}_{batch} \leftarrow$ sample from $\mathcal{D}$ 	{\color{green} // sample a minibatch triplets of size b}
\item $\mathcal{T}_{batch} \leftarrow \emptyset$  {\color{green} //  for storing the set of pairs of correct and corrupted triplets}
\item \textbf{for} $(h,l,t) \in \mathcal{D}$ \textbf{do}
\begin{description}
\item $(h^{'},\ell,t^{'}) \leftarrow$ sample from $\mathcal{D}^{-}_{(h,l,t)}$ {\color{green} // sample a corrupted triplet}
\item $\mathcal{T}_{batch} \leftarrow \mathcal{T}_{batch} \cup \{ ((h,\ell,t),(h^{'},\ell,t^{'})) \}$
\end{description}
\item \textbf{end for}
\item $\mathcal{E}_{batch} \leftarrow$  head and tail set of $\mathcal{T}_{batch}$
\item $\mathcal{R}_{batch} \leftarrow$  link set of $\mathcal{T}_{batch}$ 
\item Updates embeddings w.r.t.  
\begin{equation*}
\sum_{(h, \ell, t),(h^{'},\ell,t^{'})\in  \mathcal{T}_{batch}} {[\gamma + \mathrm{E}(h,l,t)- \mathrm{E}(h^{'},\ell,t^{'}) ]_+}  +\lambda_1 \sum_{m \in \mathcal{E}_{batch}} \left[\|\bm{e}_m\|_2^2 - 1\right]_+ + \lambda_2 \sum_{n \in \mathcal{R}_{batch}} \| \bm{r}_{n} \|_2^2,
\end{equation*}

\end{description}

\end{enumerate}

\end{algorithm*}


\subsection{Discussion}
The proposed idea is inspired by the limitation of one of the most commonly used algorithm \emph{transE} \cite{bordes_translating_2013}  in relational learning. The Energy function of this model can be expressed in the following:
\begin{equation}
\mathrm{E}{(h,\ell, t)} = \|\bm{e}_h + \bm{r}_{\ell}-\bm{e}_t\|_2^2
\end{equation}

For any triplets of the form \{$(h, r_1, t)$, $(h, r_2, t)$, \ldots, $(h, r_m, t)$\}, the \emph{TransE}  and many other algorithms may result in all the relation types  \{ $r_1$, $r_2$, \ldots, $r_m$\} involved to be equal. As in the case of \emph{TransE} algorithm, it tends to converge all the relation vectors into the subtracting vecotr  $\bm{e}_h - \bm{e}_t$.  Example triples are \emph{(Obama, presidentOf, USA)}, \emph{(Obama, placeOfbirth, USA)}.

 On the other hand, considering the simple link of ``$isa$" in  the statement \emph{Bob Dylan was a song writer, singer, performer, book author and film actor}, the following relation triples can be generated from it:
\\
\begin{tabular}[center]{l l l} 
 \\
 \emph{head} & \emph{link} & \emph{tail} \\
 \hline 
 (\emph{BobDylan}, & \emph{isa}, & \emph{SongWriter}) \\
 (\emph{BobDylan}, & \emph{isa}, & \emph{Singer}) \\
  (\emph{BobDylan}, & \emph{isa}, & \emph{Performer}) \\
   (\emph{BobDylan}, & \emph{isa}, & \emph{BookAuthor }) \\
    (\emph{BobDylan}, & \emph{isa}, & \emph{FilmActor}) \\\\
 \end{tabular}
 
Both \emph{transE} and many other existing algorithms assign one unified representation for the link $r_{\ell}$ in all triplets.  Since the learned energies of the valid triples  possess low values,  different entities such as song writer, singer and performer are  distributed very close to each other in the embedding space due to their closeness to the same embedding vector of  the link ``$isa$". Although there exist distinctions between the occupations represented by these entities, the learned embeddings are not able to highlight such distinction. To overcome this,  the proposed energy function  first projects the link embedding vector onto the hyperplane spanned by the two  entity embedding vectors, and then performs the distance computation within the hyperplane.  This subsequently results in different representations of the link when it gets involved with different entity pairs.  

The energy function  $\mathrm{E}{(h,\ell, t)} = \|\textbf{P}_l\bm{e}_h + \bm{r}_{\ell}-\textbf{P}_l\bm{e}_t\|_2^2$ of \emph{transR} \cite{lin_learning_2015} is another one proposed to overcome the limitation of   \emph{transE}. It allows the entities and links to be distributed in different embedding spaces of dimensions $d$ and $k$, respectively, and  introduces a set of $ k\times d$ projection matrices $\{\textbf{P}_l\}_{l=1}^{N_r}$ to align the two spaces over each link. In a way, it fixes an embedding space for the links, and project each pair of the entities $(\bm{e}_h, \bm{e}_t)$ onto this space to further formulate the distance. For the same entity,  its final representation $\textbf{P}_l\bm{e}_h$ used for comparing distance  differs over different links. On the contrary, the proposed method allows various customised representations $\textbf{P}_{ht} \bm{r}_{\l}$ for a link to suit the needs of different entity pairs, but maintains the fixed embedding representation for the entities when comparing distance. This thus allows the opportunity to propagate the relation information through entities. For instance, we assume there exist entities $c_1,c_2, \ldots,d_1,d_2,\ldots, e_1, e_2, \ldots$ belonging to three classes of $C,D,E$ and the class structure can be reflected by the link information.   When the entities are projected in a same fixed space,  their embeddings are able to  show directly the within-class closeness and between-class dispersion in the same space, thus it is possible to  transfer the instance-based inference to the class-based inference e.g., from $(c_i, r_1, d_j) \wedge (d_j, r_2,, e_k) \Rightarrow  (c_i, r_3, e_k)$ to $(C, r_1, D)\wedge (D, r_2, E) \Rightarrow  (C, r_3, E)$.



\section{Experiments} \label{exp}
To compare our proposed algorithm with various other methods from the literature, we evaluate our approach on the two typical knowledge base, Wordnet \cite{miller_wordnet:_1995} and Freebase \cite{bollacker_freebase:_2008}. We listed some statistics of these data in Table \ref{data}.

\subsection{Data sets}
\subsubsection{Wordnet} 

This is a large lexical database of English. It groups words into sets of cognitive synonyms (synsets) and interlink these synsets by means of a small number of semantic relations, such as \emph{synonymy}, \emph{antonymy}, \emph{meronymy} and \emph{hypernymy}.  For example, a typical triplet \emph{(\_range\_NN\_3, \_hypernym, \_tract\_NN\_1)} means the third meaning of the noun "range"  is a hypernym of the first sense of the noun "tract". We consider the data \emph{WN18} used in \cite{bordes_semantic_2014}, which contains 18 relational types.

\subsubsection{Freebase} 
Freebase is a massive online collection database consisting of general human knowledge. It organised the human knowledge data  directly in the triplet form of \emph{(head, link, tail)}. Unlike the Wordnet database, it does not disambiguate the senses of each entity. Typical triplet example includes \emph{(Peasant Magic, album, Solipsitalism)}, \emph{(Barack Obama, religion, Christianity)} and \emph{(Orange France, place-founded, Paris)}. FB15k \cite{bordes_learning_2011} is created by selecting the frequent occurrent entities and have been used for comparison with other algorithms in this paper.

\begin{table}[t]
\caption{Statistics of data sets} \label{data}
\centering
\begin{tabular}[center]{|l |l l |} 
 \hline
Dataset & WN18 & FB15k \\
 \hline 
Relationships & $18$ & $1,345$ \\
Entities & $40,943$ & $14,951$ \\
Train & $141,442$ & $483,142$ \\
Valid & $5,000$ & $50,000$ \\
Test & $5,000$ & $59,071$ \\
\hline
 \end{tabular}
 \end{table}
 
\subsection{Evaluation}
We compare the proposed algorithm with the state-of-art  methods on the WN18 and FB15k data set for link prediction. Essentially, every method trains a score function (or energy function in our case) to assemble the likely relations with a higher score (or lower energy) than the unlikely relations. This score function (energy function) corresponding to each model's trained parameters could thus gives its estimation of the likely score for every true triplets in the test set. Two evaluation metric \cite{bordes_learning_2011}, \emph{mean rank} and \emph{hits@10}, about measuring how well the predicted score matched with the true triple in the test set have been applied in our paper.

\begin{table*}
\caption{Evaluation Results on WN18 and FB15k} \label{result1}
\centering
\begin{tabular}[center]{| c | c  c  | c c | c   c | c  c |}  
 \hline
Dataset & \multicolumn{4}{c}{WN18} & \multicolumn{4}{|c|}{FB15k} \\
\hline
\multirow{2}{*}{Metric} & \multicolumn{2}{ c |}{Mean Rank} & \multicolumn{2}{c |}{Hits@10(\%)} & \multicolumn{2}{c |}{Mean Rank}  &  \multicolumn{2}{c|}{Hits@10(\%)}   \\ 
\multicolumn{1}{|  c |}{} 	& 
Raw & Filter & Raw & Filter& Raw & Filter &  Raw & Filter \\
\hline
Unstructured \cite{bordes_semantic_2014} & $315$ & $304$ & $35.3$ & $38.2$ & $1,074$ & $979$ & $4.5$ & $6.3$ \\
\hline
RESCAL \cite{nickel_three-way_2011} & $1,180$ & $1,163$ & $37.2$ & $52.8$ & $828$ & $683$ & $28.4$ & $44.1$ \\
\hline
SE \cite{bordes_learning_2011} & $1,011$ & $985$ & $68.5$ & $80.5$ & $273$ & $162$ & $28.8$ & $39.8$ \\
\hline
SME(linear) \cite{bordes_semantic_2014} & $545$ & $533$ & $65.1$ & $74.1$ & $274$ & $154$ & $30.7$ & $40.8$ \\
\hline
SME(bilinear) \cite{bordes_semantic_2014}  & $526$ & $509$ & $54.7$ & $61.3$ & $284$ & $158$ & $31.3$ & $41.3$ \\
\hline
LFM \cite{jenatton_latent_2012}  & $469$ & $456$ & $71.4$ & $81.6$ & $283$ & $164$ & $26.0$ & $33.1$ \\
\hline
TransE \cite{bordes_translating_2013} & $263$ & $251$ & $75.4$ & $89.2$ & $243$ & $125$ & $34.9$ & $47.1$ \\
\hline
TransH \cite{wang_knowledge_2014} & $318$ & $303$ & $75.4$ & $86.7$ & $211$ & $84$ & $42.5$ & $58.5$ \\
\hline
TransR \cite{lin_learning_2015} & $232$ & $219$ & $78.3$ & $91.7$ & $226$ & $78$ & $43.8$ & $65.5$ \\
\hline
CTransR \cite{lin_learning_2015} & $243$ & $230$ & $\mathbf{78.9}$ & $\mathbf{92.3}$ & $233$ & $82$ & $44$ & $66.3$ \\
\hline
\emph{TranPES} & $\mathbf{223}$ & $\mathbf{212}$ & $71.6$ & $81.3$ & $\mathbf{198}$ & $\mathbf{66}$ & $\mathbf{48.05}$ & $\mathbf{67.3}$ \\
\hline
\end{tabular}

\quad
\caption{Detailed Evaluation on FB15k} \label{result2}

\centering
\begin{tabular}{| c | c  c  | c c | c   c | c  c |} 
 \hline
Tasks & \multicolumn{4}{ c |}{Predicting Head(Hits@10)} & \multicolumn{4}{c |}{Predicting Tail(Hits@10)}  \\ 
\hline
Relation Category	& 
1-To-1 & 1-To-M & M-To-1 & M-To-M& 1-To-1 & 1-To-M &  M-To-1 & M-To-M \\
\hline
Unstructured & $34.5$ & $2.5$ & $6.1$ & $6.6$ & $34.3$ & $4.2$ & $1.9$ & $6.6$ \\
\hline
SE & $35.6$ & $62.6$ & $17.2$ & $37.5$ & $34.9$ & $14.6$ & $68.3$ & $41.3$ \\
\hline
SME(linear) & $35.1$ & $53.7$ & $19.0$ & $40.3$ & $32.7$ & $14.9$ & $61.6$ & $43.3$ \\
\hline
SME(bilinear) & $30.9$ & $69.6$ & $19.9$ & $38.6$ & $28.2$ & $13.1$ & $76.0$ & $41.8$ \\
\hline
TransE & $43.7$ & $65.7$ & $18.2$ & $47.2$ & $43.7$ & $19.7$ & $66.7$ & $50.0$ \\
\hline
TransH & $66.7$ & $81.7$ & $30.2$ & $57.4$ & $63.7$ & $30.1$ & $83.2$ & $60.8$ \\
\hline
TransR & $76.9$ & $77.9$ & $38.1$ & $66.9$ & $76.2$ & $38.4$ & $76.2$ & $69.1$ \\
\hline
CTransR & $\mathbf{78.6}$ & $77.8$ & $36.4$ & $\mathbf{68.0}$ & $77.4$ & $37.8$ & $78.0$ & $\mathbf{70.3}$ \\
\hline
\emph{TranPES} & $78.0$ & $\mathbf{88.6}$ & $\mathbf{38.9}$ & $67.3$ & $\mathbf{78.9}$ & $\mathbf{42.1}$ & $\mathbf{84.2}$ & $69.8$ \\
\hline
\end{tabular}

\end{table*}

\subsubsection{mean rank}
 For each correct triplet in the test set, we first construct the corrupt triples by replacing the head entity with all the entities in knowledge base in turn. The score (or energy) of these corrupted triples are computed by the model and have been sorted in descending (or ascending) order. The rank of the correct head entity is finally stored. This procedure is repeated by replacing the tail entity of the each triplet in the test set. The mean of all these predicted ranks  in the test set have been used as the \emph{mean rank} assessment.

\subsubsection{hits@10} 
It is very biased by only utilising the \emph{mean rank} evaluation scheme. This is because that this evaluation scheme only cares about the dominant relationship related triples if the test data is heavily unbalanced with relation types, consequently,  other relation types' contribution to the \emph{mean rank} assessment have been ignored.  Thus, the proportion of the correct entities ranked within top $10$ have been reported as a supplemental evaluation indicator. In the model training, it is important to select the margin $\gamma$ for balancing the importance of each relationship, which could result a better $hits@10$ performance. 

The ranking of each entity within these evaluation metric can be inaccurate when some corrupted triples turns out to be the correct ones in the knowledge graph. \cite{bordes_translating_2013} suggest that filtering out the corrupted triples which appears either in the training, validation or test set before ranking those entities.  We labeled the first evaluation metric as \emph{raw} and the later filtered evaluation metric as \emph{filtered}.


\subsection{Results}
In the experimental phase, we direct compare our proposed models with those results reported in the literature  \cite{lin_learning_2015} as we are using the same data sets. For the experiments of the \emph{TranPES} algorithm, we first identify the scale of each parameters by searching the learning rate among $\{ 0.1, 0.01, 0.002\}$, the dimensions of the entity and link embedding $k$ among $\{ 20, 50, 100 \}$, the scaling control factor $\lambda_1$ is assigned as a constant value $1$, batchsize $B$ among $\{50, 100, 200\}$, the regularisation parameter $\lambda_2$ among $\{ 1,$ $0.1,$ $0.01\}$ and the margin $\gamma$ among $\{0.5, 0.6, 0.7, 0.8, 0.9, 1.0\}$. The value of margin $\gamma$ is closely related to the $hits@10$ evaluation metric thus it is important to do a full searching for this parameter. For both datasets, we limit the epochs round at most 1000 times.  And we select the best model according to the best mean rank on the validation set. An open-source implementation of \emph{TranPES} is availabe from the project webpage\footnote{\url{https://github.com/while519/TranPES.git}}.

The optimal configuration were: $k=20, B = 100, \lambda_2=0.01, \gamma=0.7$ on WN18 and $k=100, B=100, \lambda_2 = 0.01, \gamma=0.4$ on FB15k. The evaluation results are present on Table. \ref{result1}.

On the WN18 data set, it seems that the proposed algorithm does not perform better than even its simpler form of TransE algorithm. While on the more complex data set FB15k, the \emph{TranPES} outperforms all the other algorithms.  The performances of \emph{TransR} and \emph{TranPES} is close on the FB15k, but the \emph{TranPES} algorithm has much less parameters for the training(only $O(N_ek+N_rk)$).

We followed the same detailed evaluation protocal proposed in \cite{bordes_translating_2013}. It classifies the $hits@10$ results according to the four categories each relationship belong to:  1-To-1, 1-To-Many, Many-To-1, Many-To-Many. For instance, a given relationship is classified in to 1-To-Many if a \emph{head} can appear with at many \emph{tails} and a \emph{tail} can appear with at most one \emph{head}.  The corresponding result was shown on Table. \ref{result2}. 

As one would expect, it is approachable to predict \emph{head} in the 1-To-1, 1-To-Many  relationships and predict \emph{tail} in the 1-To-1, Many-To-1 relationship. This has been shown on the \emph{TranPES} results subsequently. Upon successfully predicting these relationships, it shows the reasoning ability of \emph{TranPES} algorithm.




\section{Conclusion and Future Work} \label{conclusion}
We present a relational latent feature model for embedding the objects (e.g. \emph{relations}, \emph{entities}) of the multi-relational data into a semantic vector space.  The operation (translation, projection .etc) between the vectors in this space has then been used to model the interactions within the multi-relational data. This developed algorithm, named \emph{TranPES}, can be viewed as an extension of the existing TransE algorithm. Comparing our approach with the existing algorithms on the complex FB15k knowledge database indicate that the \emph{TranPES} algroithm finds a better trade-offs between model complexity and model accuracy.

Our idea was developed to amend the inappropriate assumptions of the simple TransE model to deal with some common pattern within the knowledge graph. For example, the simple triplets of \emph{(Obama, bornIn, USA)} and \emph{(Obama, presidentOf, USA)} would result in the collapse of the TransE's assumption, however, we still can express such triplets by modelling the interact in the way of our method.

In the experiment, we have showed the powerful of the relational learning model for the link prediction tasks, specially, the 1-To-1, 1-To-Many, Many-To-1 relation types. But the successful of the reasoning mechanism within the knowledge graph has not been showed. Future work might study to learn the latent feature model as well as giving clues on how do the other triplets in the knowledge base contribute on the link prediction task.

\section*{Acknowledgment}
This research has been supported by the China Scholarships Council.

\bibliographystyle{abbrv}
\bibliography{My_Library}

\balancecolumns

\end{document} 